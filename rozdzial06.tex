\chapter{Podsumowanie}
\section{Uzyskane efekty pracy}
Celem niniejszej pracy była ewaluacja wydajności interfejsów programowania aplikacji implementowanych w dwóch różnych technologiach, w odniesieniu do licznych aspektów dotyczących sposobów ich wykorzystania. Zdecydowano się nie tylko na analizę efektywności podstawowych rodzajów operacji protokołu hipertekstowego, wchodzących w skład tak popularnych dzisiaj usług sieciowych, ale także wykorzystania mechanizmów programowania współbieżnego, technik obsługi żądań asynchronicznych, implementacji zaawansowanego wzorca projektowego, zastosowania rozwiniętych technik optymalizacji pozyskiwania danych, a także wdrożenia rozwiązań w kontekście środowisk chmurowych.

Zwrócenie uwagi na tak wiele aspektów dotyczących interfejsów programowania aplikacji miało na celu uświadomienie czytelnika, że tego typu systemy internetowe, wykorzystywane są powszechnie nie tylko do realizacji najpopularniejszych czterech, podstawowych operacji na danych. Wachlarz możliwości związanych z tworzeniem internetowych API jest znacząco szerszy, a fakt ten uwydatnia się wraz z rosnącym poziomem skomplikowania usług sieciowych, a także zadań, które są przed nimi stawiane.

Zastosowanie mnogości kontekstów, w których odnaleźć musiały się przygotowane rozwiązania miało również odmienny cel. Misją autora było dowiedzenie się czy którykolwiek z systemów opartych o dwie porównywane technologie wdrożeniowo-uruchomieniowe, wykazuje wysoką wydajność względem swojego konkurenta w którymkolwiek z obszarów prowadzonych badań. Jeżeli tak, to które z tych obszarów są faworyzowane przez konkretne technologie.

Wyniki przeprowadzonych badań umożliwiły rozwianie powyższych wątpliwości, a także uzyskanie dodatkowej wiedzy, która nawet dla osób posiadających doświadczenie w zakresie kompozycji oraz tworzenia interfejsów API nie musi wydawać się oczywista. Zrealizowane eksperymenty uwidoczniły niektóre zależności, zadając innym z kolei kłam. Przykładem potwierdzenia spodziewanej hipotezy, może być wykazana wyższość wydajności rozwiązań implementowanych na platformach dedykowanych względem platform generycznych. Kolejną egzemplifikacją, w ramach której, jeszcze przed przeprowadzeniem badania, sformułować można było silną hipotezę, była obserwacja wpływu zastosowania usprawnień wydajnościowych, separacji środowisk bazodanowych, a także wdrożenia wzorca podziału odpowiedzialności. Rezultaty badania systemów bazodanowych z kolei, mogą być doskonałym argumentem, na obalenie hipotezy wyższej efektywności komunikacji silników baz danych oraz interfejsów tworzonych na bazie technologii jednego producenta.

Odnosząc się do dodatkowej wiedzy, której chęć pozyskiwania wzmożona została poprzez ambicję wyjaśnienia pojawiających się w badaniach anomalii, wspomnieć należy o sposobie obsługi wielowątkowej w odniesieniu do współbieżnie generowanych, długotrwających żądań. Obsługa ta niemalże nie występuje w kontekście interfejsu języka JavaScript, natomiast jest wydatnie rozbudowana w przypadku usługi implementowanej w C\# i uruchamianej na platformie .NET. Ponadto ciekawym jest również fakt, jak bardzo prostota, tycząca się mechanizmów wywoływania operacji asynchronicznych, może nieść korzyść dotyczącą wydajności ich realizacji.

W niektórych przypadkach jednak konwencjonalność rozwiązania nie idzie w parze z jego wydajnością. Potwierdzeniem tego właśnie stwierdzenia są przeprowadzone badania dotyczące podstawowego oraz autorskiego podejścia do realizacji mechanizmów pamięci podręcznej. W ramach pracy tej, zaimplementowano, a także zbadano zachowanie systemu cache uwzględniającego częstotliwość wywoływania punktu końcowego, a także liczbę unieważnień identyfikującego go wpisu. Zgromadzone rezultaty należy postrzegać jako obiecujące, jednakże wymagana jest zdecydowanie bardzej obszerna analiza uwzględniająca zmienność liczby momentów unieważnień, czy też wpływ dysproporcji parametrów w różnych chwilach obsługi żądań.

Wspomnieć należy również o przeprowadzonych w ramach niniejszego badania parowych testach statystycznych, które pozwoliły na wykazanie statystycznie istotnej przewagi określonych konfiguracji rozwiązań względem pozostałych z nich. 

Bardzo ważnym jest również uwypuklenie pewnej tezy. Przeprowadzony zestaw badań nie wskazał, jednakże przede wszystkim nie miał wskazać, technologii niezaprzeczalnie lepszej. Technologia taka nie istnieje, a wynika to w głównej mierze z ilości obszarów, w kontekście których może ona zostać wykorzystana oraz badana. Dlatego też, dokument ten, może okazać się pomocny dla tych osób, którzy zainteresowani są oceną poziomu wydajności interfejsu dla konkretnej technologii oraz konkretnego sposobu jej wykorzystania.
\section{Perspektywy rozwoju badań}
Każdy z przytoczonych obszarów wykorzystania interfejsów programowania aplikacji, reprezentowany w niniejszej pracy poprzez odmienne badanie, może zostać z łatwością rozbudowany poprzez ewaluację dodatkowych metryk wydajnościowych, czy też zmianę konfiguracji środowiska badawczego. Zdecydowano się na wskazanie perspektyw rozwoju badań w odniesieniu do tych domen funkcjonalności API, które zostały poruszone w tym dokumencie.

Odwołując się do badania wpływu wykorzystania systemów bazodanowych w kontekście porównywanych technologii, jako perspektywę rozwoju wskazać należy przeprowadzenie ewaluacji wydajności dla przedziałów liczby wątków-generatorów o zmiennej długości. Ponadto, wzięte pod uwagę mogą być również te spośród systemów bazodanowych, w ramach których nie dostarczane jest wsparcie dla mapperów obiektowo-relacyjnych Entity Framework Core oraz Prisma.

W ramach badania realizacji operacji współbieżnych, wprowadzić można dodatkowe rodzaje algorytmów metaheurystycznych dla różnych problemów o wysokiej złożoności obliczeniowej. Interesującą perspektywą rozwoju tego badania, jest również implementacja odmiennych heurystyk dla symetrycznego problemu komiwojażera, a także porównanie ich wykonania dla interfejsów wspierających przetwarzanie wielowątkowe.

Badanie wydajności operacji asynchronicznych może zostać poszerzone o uwzględnienie różnych implementacji klientów protokołu hipertekstowego, a także konfiguracji poszczególnych ich parametrów.

Najwięcej perspektyw rozwoju badań, wyróżnić należy w kontekście ewaluacji porównywanych mechanizmów pamięci podręcznej. Zbadane mogą zostać między innymi odmienne metryki wpływające na zmianę czasu odpowiedzi na żądanie. Jako przykładową metrykę wskazać można narzut wydajnościowy wprowadzany przez metodę kalkulacji czasu życia wpisu w magazynie pamięci podręcznej. Ponadto, struktura wykonanego badania, mogłaby zostać dostosowana względem zmiennego czasu trwania testu, zmiennego natężenia ruchu sieciowego, czy też deterministycznego charakteru wywoływania żądań unieważniających. Co więcej, zaproponowana przez autora metoda może zostać zmodyfikowana poprzez uzależnienie czasu przechowywania wpisu od dodatkowych parametrów, bądź też zmianę ich istotności względem wyliczania czasu życia elementu pamięci cache.

W kontekście wprowadzenia wzorca podziału odpowiedzialności, a także separacji modeli bazodanowych, badanie może zostać poszerzone o zastosowanie odmiennych mechanizmów replikacji, niż wykorzystana w tej pracy technika transakcyjna. W takim przypadku, należy zbadać w jakim czasie, od momentu dodania rekordu bazodanowego, będzie on dostępny w ramach źródła danych przeznaczonego do odczytu. Co więcej, należy pamiętać, że celem zdefiniowania specyficznej struktury obsługi żądania, była możliwość optymalizacji modeli danych. Dlatego też, interesującą perspektywą rozwoju tego badania, mogłoby być wskazanie, które z zaimplementowanych technik optymalizacji mają kluczowe znaczenie pod kątem wydajności, a które wpływają na nią jedynie nieznacznie.

Ostatnie ze zidentyfikowanych perspektyw rozwoju tyczą się obserwacji efektywności działania w odniesieniu do produkcyjnych środowisk chmurowych. Biorąc pod uwagę ten właśnie aspekt, możliwym jest dokonanie porównania dla większej liczby platform wdrożeniowych, czy też zbadanie, w jaki sposób usługi generyczne typu infrastructure-as-a-service mogą zostać zoptymalizowane, aby implementowane wewnątrz nich systemy internetowe, osiągały wydajność przybliżoną, lub wyższą względem rozwiązań dedykowanych. 
