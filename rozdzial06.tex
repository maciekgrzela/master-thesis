\chapter{Podsumowanie}
% Przykładowe zakończenie do pracy magisterskiej z bezpieczeństwa
% Celem niniejszej pracy było zweryfikowanie tego jak przedstawia się autorytet policjanta w opinii społecznej. W teoretycznej części pracy omówiono definicje autorytetu, jako najważniejszy elementem, wspólny dla większości definicji wskazano specyficzny rodzaj szacunku, gdyż posiadanie go wiąże się ze zdobyciem zaufania u drugiego człowieka.

% W początkowych rozdziałach skupiono się na roli policjanta i komponentach wpływających na jego wizerunek. W czasie wykonywania swoich obowiązków, funkcjonariusz policji musi mierzyć się z różnymi sytuacjami, jego praca często polega na rozwiązywaniu konfliktów, dlatego tak istotna jest odpowiednia komunikacja. Nawiązywanie bezpośrednich relacji z członkami społeczeństwa wymaga, żeby zdolności w zakresie porozumiewania się były na najwyższym poziomie. Następnym czynnikiem, który wpływa na wyobrażenie na temat policjantów jest ich wyszkolenie. Jeśli obywatel zaobserwuje, iż policjant nie ma wiedzy czy też umiejętności w zakresie swoich obowiązków, bez wątpienia niekorzystnie wpłynie to na autorytet policjanta. Dodatkowo badania pokazały, że media mają duże znaczenie w kreowaniu wizerunku funkcjonariusza, środki masowego przekazu w założeniu, chcąc uzyskać jak najlepszą oglądalność, skupiają się na wydarzeniach szokujących, budzących kontrowersje. W telewizji, czy w radiu mówi się o policjantach, którzy nadużywają swoich uprawnień, nie wykonali odpowiednio swoich obowiązków. Takie komunikaty trafiające do ludności stawiają funkcjonariuszy w złym świetle.

% Wyniki przeprowadzonych badań umożliwiły udzielenie odpowiedzi na problem badawczy niniejszej pracy postawiony w formie pytania: „Jaki autorytet posiada w dzisiejszych czasach funkcjonariusz policji?”. Badania wykazały, iż większość respondentów nie jest w stanie określić osobistego stanowiska w zakresie zaufania do analizowanej formacji. Powodem dla tego stanu rzeczy może być ich dystans do organizacji bądź brak świadomości tego jak ważną funkcję pełnią policjanci w ich życiu codziennym. Niewielka część badanych przyznała, że ufa Policji, dużo większy procent odpowiedział, że ma zastrzeżenia i nie może zadeklarować pełnego zaufania, co wskazuje na konieczność przeprowadzenia zmian na wielu płaszczyznach. Można przykładowo zorganizować szkolenia dla funkcjonariuszy, które wpłyną na skuteczność podejmowanych działań a w rezultacie przyczynią się do zwiększenia autorytetu u obywateli.

% Funkcjonariusze Policji w Polsce nieustannie starają się zbudować nowy etos służby, przez co aktualnie Policja jest dużo bardziej cenioną instytucją niż działająca wcześniej Milicja Obywatelska. Czynności wykonywane na rzecz społeczeństwa spowodowały, że Policja jest rozpatrywana jako organizacja służebna. Zaznacza się, iż formacja chcąc zdobyć autorytet wśród ludzi powinna być uważna na zgłaszane przez nich potrzeby, ponieważ jedynie wtedy zapracuje na zaufanie społeczeństwa, które jest podstawą w budowaniu korzystnego wyobrażenia na temat całej formacji jak i funkcjonariuszy policji.
\section{Wnioski z przeprowadzonych badań}
\section{Perspektywy rozwoju badań}
