\chapter{Podsumowanie}
\section{Uzyskane efekty pracy}
Celem niniejszej pracy była ewaluacja wydajności interfejsów programowania aplikacji implementowanych w dwóch różnych technologiach, w odniesieniu do licznych aspektów dotyczących sposobów ich wykorzystania. Zdecydowano się, nie tylko na analizę efektywności podstawowych rodzajów operacji protokołu hipertekstowego, wchodzących w skład tak popularnych dzisiaj usług sieciowych, ale także wykorzystania mechanizmów programowania współbieżnego, technik obsługi żądań asynchronicznych, implementacji zaawansowanego wzorca projektowego, zastosowania rozwiniętych technik optymalizacji pozyskiwania danych, a także wdrożenia rozwiązań w kontekście środowisk chmurowych.

Zwrócenie uwagi na tak wiele aspektów dotyczących interfejsów programowania aplikacji miało na celu uświadomienie czytelnika, że tego typu systemy internetowe, wykorzystywane są powszechnie nie tylko do realizacji najpopularniejszych czterech, podstawowych operacji na danych. Wachlarz możliwości związanych z tworzeniem internetowych API jest znacząco szerszy, a fakt ten uwydatnia się wraz z rosnącym poziomem skomplikowania usług sieciowych, a także zadań, które są przed nimi stawiane.

Zastosowanie mnogości kontekstów, w których odnaleźć musiały się były przygotowane rozwiązania, miało też odmienny cel. Misją autora było dowiedzenie się czy którykolwiek z systemów opartych o dwie porównywane technologie wdrożeniowo-uruchomieniowe, wykazuje wysoką wydajność względem swojego konkurenta w którymkolwiek z obszarów prowadzonych badań. Jeżeli tak, to które z tych obszarów są faworyzowane przez konkretne technologie.

Wyniki przeprowadzonych badań umożliwiły rozwianie powyższych wątpliwości, a także uzyskanie dodatkowej wiedzy, która nawet dla osób posiadających doświadczenie w zakresie kompozycji oraz tworzenia interfejsów API, nie musi wydawać się oczywista. Zrealizowane eksperymenty uwidoczniły niektóre zależności, zadając innym z kolei kłam. Przykładem potwierdzenia spodziewanej hipotezy, może być wykazana wyższość wydajności rozwiązań implementowanych na platformach dedykowanych względem platform generycznych. Rezultaty badania systemów bazodanowych z kolei, mogą być doskonałym argumentem, na obalenie hipotezy wyższej efektywności komunikacji silników baz danych oraz interfejsów tworzonych na bazie technologii jednego producenta.

Odnosząc się do dodatkowej wiedzy, której chęć pozyskiwania wzmożona została poprzez ambicję wyjaśnienia pojawiających się w badaniach anomalii, wspomnieć należy o sposobie obsługi wielowątkowej w odniesieniu do współbieżnie generowanych, długotrwających żądań. Obsługa ta, niemalże nie występuje w kontekście interfejsu języka JavaScript, natomiast jest wydatnie rozbudowana w przypadku usługi implementowanej w C\# i uruchamianej na platformie .NET. Ponadto, ciekawym jest również fakt, jak bardzo prostota, tycząca się mechanizmów wywoływania operacji współbieżnych, może nieść korzyść dotyczącą wydajności ich realizacji.

W niektórych przypadkach jednak, prostota rozwiązania nie idzie w parze z jego wydajnością. Potwierdzeniem tego właśnie stwierdzenia są przeprowadzone badania dotyczące podstawowego oraz autorskiego podejścia do realizacji mechanizmów pamięci podręcznej. W ramach pracy tej, zaimplementowano, a także zbadano zachowanie systemu cache uwzględniającego częstotliwość wywoływania punktu końcowego, a także liczbę unieważnień identyfikującego go wpisu. Zgromadzone rezultaty należy postrzegać jako obiecujące, jednakże wymagana jest zdecydowanie bardzej obszerna analiza uwzględniająca zmienność liczby momentów unieważnień, czy też wpływ dysproporcji parametrów w różnych chwilach obsługi żądań.

Bardzo ważnym jest również uwypuklenie pewnej tezy. Przeprowadzony zestaw badań nie wskazał, jednakże przede wszystkim nie miał wskazać, technologii niezaprzeczalnie lepszej. Technologia taka nie istnieje, a wynika to w głównej mierze z ilości aspektów, w kontekście których może ona zostać wykorzystana. Dlatego też, dokument ten, może okazać się pomocny dla tych osób, którzy zainteresowani są poziomem wydajności interfejsu dla konkretnej technologii oraz konkretnego sposobu jej wykorzystania. Czytelnik nie znajdzie tu odpowiedzi, na pytanie dotyczące ogólnej wyższości któregokolwiek z porównywanych rozwiązań.  
\section{Perspektywy rozwoju badań}
