\chapter{Opis problemu badawczego}
W ramach niniejszego rozdziału omówiony został podjmowany problem badawczy. W związku z jego złożonością, autor pracy zdecydował się na podział tego zagadnienia na określone aspekty, będące różnorodnymi względem funkcjonalności internetowych interfejsów programowania aplikacji. Na koniec tej części pracy, na podstawie sformułowanych konktekstów badawczych, zdefiniowano listę scenariuszy realizacji badań.
\section{Przedstawienie aspektów problemu badawczego}
Podejmowany w ramach niniejszej pracy problem badawczy, tyczy się wydajności usług sieciowych, których charakterystyka wyróżnia zastosowanie wielu odrębnych komponentów programistycznych. Komponentami tymi, mogą być zarówno: biblioteki obsługujące proces mapowania obiektowo-relacyjnego, zastosowany wzorzec projektowy w kontekście wewnętrznej architektury API, rodzaj wykorzystywanego zewnętrznego źródła danych, czy też wybrana chmurowa platforma wdrożeniowa. W związku z mnogością zagadnień występujących w ramach niniejszego problemu badawczego, zdecydowano się na dokonanie podziału jego opisu w taki sposób, aby na podstawie każdego z aspektów, możliwe było zdefiniowanie wyspecyfikowanych scenariuszy realizacji badań.

Każdy z aspektów problemu badawczego wymieniony poniżej, rozpatrywany będzie względem dwóch odrębnych zestawów technologii programistycznych (tj. C\#/.NET oraz JavaScript/NodeJS). Scenariusze badawcze, opracowane na podstawie, każdego z opisanych w tym podrozdziale aspektów, uwzględniać będą porównanie uzyskanych wyników badań dla obu wymienionych rozwiązań informatycznych.
\subsection*{Wydajność interfejsu API względem liczby żądań generowanych przez klientów}
Pierwszym z omawianych aspektów rozważanego problemu badawczego jest wpływ wydajności działania interfejsu programowania aplikacji, względem liczby klientów, którzy w sposób równoległy generują żądania w kierunku API.

Wydajność, w kontekście tego właśnie aspektu, interpretowana jest poprzez metryki czasu odpowiedzi na żądanie, a także procentowe wartości wykorzystania zasobów sprzętowych interfejsu programowania aplikacji, takich jak centralna jednostka przetwarzania, czy też pamięć operacyjna o dostępie swobodnym.

Zgodnie z obowiązującymi praktykami realizacji pomiarów wydajności usług sieciowych, charakterystyka ta, powinna być wyliczana w oparciu o technikę testowania rozproszonego \textit{(ang. Distributed Testing)}. Przedstawiana technika, zakłada wykorzystanie wielu odrębnych systemów informatycznych, wykonujących równolegle ewaluację obciążeniową. Wartości metryk, uzyskane dla każdej z maszyn przeprowadzających testy, powinny zostać zgrupowane, a także analizowane jako pochodzące z jednego źródła.

Zmiana wydajności, obserwowana powinna być wraz ze stałym zwiększeniem natężenia liczby klientów, a dla każdego z kolejnych przedziałów liczbowych, uwzględniających kolejne przyrosty wysyłanych pakietów, metryki wydajnościowe powinny być porównywane względem ustalonego wskaźnika referencyjnego. Kalkulacja tego wskaźnika natomiast, wykonywana jest w oparciu o ewaluację wydajności dla standardowych warunków pracy interfejsu programowania aplikacji. Przykładem takich warunków, może być realizacja testu obciążeniowego dla pojedynczej maszyny testującej.

Jednym z podstawowych, otwartych standardów, które posłużyć mogą do budowy wskaźnika referencyjnego jest APDEX \textit{(ang. Application Performance Index)}. Indeks ten, pozwala na zdefiniowanie trzech przedziałów liczbowych, określających odczucia klienta testującego oprogramowanie. Przedziały te, przedstawiane są jako progi satysfakcji, tolerancji oraz frustracji. Po wykonaniu testów odbywających się w standardowych warunkach pracy usługi sieciowej, ustalenie wartości metryk wydajnościowych dla każdego z trzech progów jest możliwe, a co za tym idzie, możliwe jest także porównanie wyników uzyskiwanych przy dowolnej liczbie klientów API, względem ustalonych progów.

Na podstawie weryfikacji wydajności interfejsu API względem liczby generowanych równolegle żądań, należy także ustalić graniczną wartość sumy maszyn klienckich, dla których interfejs programowania aplikacji jest w stanie obsługiwać zapytania, w czasie zawierającym się w każdym z przedziałów referencyjnych. 
\subsection*{Korelacja charakterystyk wydajnościowych względem określonego zewnętrznego źródła danych}
Kolejny z kontekstów problemu badawczego dotyczy wpływu zastosowania odmiennych zewnętrznych źródeł danych, na efektywność pracy usługi sieciowej jaką jest interfejs API.

Koncepcja wydajności w przypadku tego aspektu problemu, rozumiana jest w sposób analogiczny do pojęcia, wprowadzonego w ramach poprzedniej sekcji.

W związku z istotnością uwzgladnienia zewnętrznych źródeł danych, jako elementów z którymi nieustannie komunikują się nowoczesne usługi sieciowe, w ramach niniejszego aspektu omawianego problemu badawczego, weryfikowany jest wpływ sposobu obsługi najpopularniejszych spośród dostępnych nieodpłatnie systemów bazodanowych, na efektywność operacji realizowanych przez poszczególne interfejsy programowania aplikacji. Problem badawczy, uwzględnia zastosowanie zarówno czterech relacyjnych systemów baz danych, jak i jednego nierelacyjnego. 

Sposób obserwacji zmiany wydajności usługi sieciowej, również jest analogiczny, do tego, który został przedstawiony w ramach poprzedniego aspektu, jednakże należy zwrócić uwagę, na konieczność ponownego zdefiniowania przedziałów satysfakcji, tolerancji oraz frustracji dla wskaźnika referencyjnego. Referencja do wartości tego wskaźnika, obliczonego bez uwzględnienia połączenia z systemem bazodanowym, prowadziła by do zaniżenia wartości ogólnej wydajności testowanego oprogramowania.

Kluczowym czynnikiem omawianego aspektu problemu badawczego jest zapewnienie deterministycznego charakteru stanu łącza sieciowego, występującego pomiędzy serwerem bazodanowym a usługą interfejsu programowania aplikacji. Dlatego też, w przypadku testów realizowanych w środowisku lokalnym, oba komponenty programowo-sprzętowe powinny znajdować się w tej samej lokalizacji. Należy podkreślić jednak, że ważnym jest, aby obie usługi informatyczne nie były uruchomione w ramach tego samego środowiska sprzętowego. Dzięki temu, pozyskane wartości metryk wykorzystania zasobów sprzętowych nie będą obarczone niedokładnością. W kontekście realizacji badań w środowisku chmurowym natomiast, ważnym jest wdrożenie zarówno API, jak i serwera bazy danych, w ramach tego samego centrum obliczeniowego, a także rozdzielenie obu usług, pomiędzy odmienne fizyczne urządzenia. Ponadto, modyfikacji powinien ulec jeden z elementów kryterium wydajności, który definiowany jest jako czas odpowiedzi interfejsu na żądanie klienta. W związku z dyspersją geograficzną obu stron komunikacji, niemożliwym jest zachowanie przewidywalnego charakteru łącza sieciowego wykorzystywanego do transmisji danych. Twierdzenie to, implikuje konieczność realizacji pomiaru czasu działania usługi sieciowej w sposób odmienny. Kryterium czasu odpowiedzi na żądanie, rozumiane w tym przypadku jest jako przedział czasowy od momentu pozyskania żądania, do momentu zakończenia wszystkich operacji, realizowanych w kontekście tego żądania. 
\subsection*{Efektywność realizacji złożonych obliczeń oraz wsparcia dla programowania współbieżnego i metod asynchronicznych}
Niniejszy aspekt problemu badawczego dotyczy wpływu wykorzystania, dostępnych w ramach określonego języka mechanizmów programowania współbieżnego, a także sposobu realizacji operacji asynchronicznych, na efektywność przeprowadzania kalkulacji w obrębie warstwy logiki biznesowej interfejsu programowania aplikacji.

Pojęcie efektywności dokonywanych kalkulacji postrzegane jest poprzez liczbę wykonanych iteracji głównej pętli zaimplementowanego algorytmu metaheurystycznego, rozwiązującego określony problem z rodziny NP-trudnych.

W celu zachowania rzetelności badań, omawiany fragment problemu badawczego uwzględnia zastosowanie analogicznego algorytmu, realizującego operacje w ten sam sposób, a także rozwiązującego ten sam problem obliczeniowy. W tym przypadku, zaobserwować będzie można fakt przystosowania technologii poddawanej ewaluacji, do dokonywania procesu zrównoleglania obliczeń, a także przeprowadzania wewnętrznej optymalizacji określonych linii zdefiniowanego kodu źródłowego.

Zaimplementowany algorytm metaheurystyczny, dostępny będzie bezpośrednio z poziomu punktów końcowych badanych interfejsów programowania aplikacji, a liczba iteracji głównej pętli algorytmu, mierzona będzie dla ustalonego, stałego czasu wykonania programu.

Ponadto, omawiany aspekt badawczy dotyczy także weryfikacji wydajności w kontekście zastosowania metod asynchronicznych. W związku ze znacząco odmienną strukturą badanych środowisk uruchomieniowych oraz języków programowania, mechanizmy obsługi operacji asynchronicznych zaimplementowane są w tych technologiach, na różnych poziomach obsługi programu. W jedym przypadku, obsługa operacji tych, jest wykonywana bezpośrednio w ramach języka programowania, natomiast w kontekście drugiej z technologii, metody których wynik nie jest dostarczany natychmiastowo, muszą zostać obsłużone wewnątrz środowiska uruchomieniowego.

Badanie weryfikacji wydajności dla funkcji asynchronicznych, oparte jest o odwołanie się interfejsu API, do współpracującej z nim hipertekstowej usługi sieciowej, pełniącej rolę pośrednika w dostępie do zdefiniowanych wewnątrz niej informacji. Przedstawiona usługa sieciowa, zostanie zaimplementowana jako odrębne oprogramowanie i będzie niezależna od obu porównywanych technologii.

W kontekście drugiej z części aspektu problemu badawczego, metryką wydajności będzie czas odpowiedzi interfejsu na żądanie.
\subsection*{Wpływ zastosowania wzorca projektowego podziału odpowiedzialności na efektywność realizacji operacji bazodanowych}
Rozważany aspekt problemu badawczego tyczy się wpływu implementacji optymalizacji wydajnościowych w kontekście komunikacji interfejsu programowania aplikacji z zewnętrznym źródłem danych.

Wykorzystując konwencjonalną trójwarstwową architekturę interfejsu API, stosowany zostaje ten sam model danych, zarówno do operacji odczytu jak i zapisu. Powoduje to brak możliwości dostosowania modelu, względem specyfiki konkretnego rodzaju operacji. Wprowadzenie wzorca projektowego separacji zapytań oraz komend ma na celu umożliwienie dokonania optymalizacji wydajnościowych wyizolowanych fragmentów modelu danych, a także wykorzystywanie ich tylko i wyłącznie w kontekście jednego typu operacji.

Co więcej, optymalizacja może być wykonana nie tylko na poziomie modelu danych, ale także w ramach fizyczynych struktur zawartych wewnątrz obsługiwanego systemu bazodanowego. Dlatego też, przedstawiany aspekt problemu badawczego dotyczy zastosowania zarówno odrębnych modeli danych wewnątrz API, odrębnych struktur programistycznych obsługujących dane modelu, jak i odseparowanych zewnętrznych źródeł danych.

Oba zastosowane źródła danych, powinny cechować się taką samą strukturą, jednakże każde z nich powinno wprowadzać charakterystyczne dla typu wykonywanych operacji, usprawnienia wydajnościowe. Ponadto, aby zachować spójność zawartości dostępnej dla klienta w ramach API, po odwołaniu się do systemu bazodanowego w celu zapisania rekordu, musi on zostać następnie zreplikowany do źródła danych obsługującego operację odczytu.

Wydajność, rozumiana poprzez czas odpowiedzi interfejsu API na żądanie klienta, powinna zostać porównana z tą, wykazywaną przez usługę sieciową opierającą się na architekturze 3-warstwowej i wykorzystującą pojedyncze źródło danych.
\subsection*{Wpływ zastosowania mechanizmów pamięci podręcznej na wydajność interfejsów API}
Problem badawczy w kontekście wykorzystania mechanizmów pamięci podręcznej, dotyczy porównania efektywności działania interfejsów programowania aplikacji implementujących standardowy oraz autorski mechanizm przechowywania rezultatów wykonanych uprzednio żądań.

Standardowy mechanizm przechowywania żądań w ramach pamięci podręcznej uwzględniać powinien stały czas ważności pojedynczego wpisu, a także jego unieważnienie w przypadku wykonania operacji modyfikującej dane. W takim przypadku, czas odpowiedzi na żądanie powinien być zwiększony w momencie konieczności odwołania się API do zewnętrznego źródła danych, a następnie zredukowany w przedziale czasowym, w ramach którego wpis pamięci podręcznej jest aktywny.

Zaimplementowany autorski mechanizm pamięci podręcznej wyróżniać będzie się zmiennym czasem ważności poszczególnych wpisów, który zależny będzie od prawdopodobieństwa wywołania określonego punktu końcowego, na podstawie informacji o liczbie historycznych wywołań. Czym większe istnieje prawdopodobieństwo ponownego wywołania punktu końcowego, tym czas ważności rezultatu przechowywanego w pamięci podręcznej będzie większy. Analogicznie do standardowego mechanizmu pamięci podręcznej, operacja modyfikacji danych unieważnia wszystkie spośród wpisów, które odwołują się do przekształconych informacji.

Celem niniejszego aspektu badawczego w ramach rozważanego problemu jest porównanie zmiany wydajności działania API, postrzeganej jako średni czas odpowiedzi na żądanie w ustalonym, stałym przedziale czasowym. Porównywane zostaną mechanizmy standardowy oraz autorski, odrębnie dla każdej z technologii programistycznych.
\subsection*{Wpływ wdrożenia interfejsu API na dedykowanej platformie chmurowej na jego efektywność działania}
Ostatni z przedstawianych aspektów problemu badawczego odnosi się do wpływu wydajności pracy interfejsu API, w zależności od rodzaju zatosowanej platformy chmurowej, na jakiej zostanie on wdrożony.

Interfejs programowania aplikacji jest usługą, której funkcjonowanie jest nieodłącznie powiązane z serwerem internetowym. Serwer sieci Web pełni rolę warstwy opakowującej, wewnątrz której działać może interfejs API. Wdrożenie rozważanej usługi sieciowej w ramach sieci Internet, wiąże się w związku z tym z uruchomieniem serwera sieci web w ramach komputera eksponowanego w sieci rozległej. Należy również zauważyć konieczność zastosowania hipertekstowego serwera pośredniczącego, po to, aby klient usługi, mógł się z nią komunikować z wykorzystaniem protokołu HTTP.

System informatyczny, składający się z przedstawionych powyżej komponentów może zostać uruchomiony w ramach wirtualnego serwera prywatnego, udostępnianego przez określonego dostawcę infrastruktury serwerowej. Model taki, definiowany jest jako infrastruktura w postaci usługi klienckiej \textit{(ang. Infrastructure as a Service)}. Ponadto, przygotowane oprogramowanie może zostać wdrożone na dedykowanej określonej technologii, platformie chmurowej. W ramach platformy tej, użytkownik, za pomocą dostarczonego interfejsu komunikacji może wdrażać oraz konfigurować działanie swojego oprogramowania. Taki model dostarczania zasobu z kolei, nazywany jest platformą w postaci usługi klienckiej \textit{(ang. Platform as a Service)}.

Niniejszy aspekt problemu badawczego, dotyczy porównania wydajności API, w zależności od jego wdrożenia na generycznym wirtualnym serwerze prywatnym, a także dedykowanej platformie chmurowej, dostosowanej pod kątem określonego środowiska uruchomieniowego oraz języka programowania.

Wskaźnik ewaluacji efektywności działania interfejsu programowania aplikacji, obejmuje te same metryki, które przedstawione zostały w pierwszym spośród omawianych aspektów problemu badawczego. Kryterium czasu odpowiedzi na żądanie, musi zostać jednakże uniezależnione od niedeterministycznego charakteru łącza internetowego, dlatego też, ten właśnie parametr, będzie dotyczył czasu od momentu otrzymania żądania przez API, do chwili wygenerowania odpowiedzi na żądanie.
\section{Sformułowanie scenariuszy badawczych}
Na podstawie przedstawionych w poprzednim podrozdziale aspektów problemu badawczego, sformułowane zostały konkretne scenariusze badawcze. W każdym ze scenariuszy, zdefiniowano zbiór czynności wykonywanych w ramach określonego badania, wyszczególniono kryteria porównawcze dla danej obserwacji, wymieniono dostosowywalne parametry badania, a także skonkretyzowano czynności, które muszą zostać podjęte, jako warunki konieczne przed wykonaniem badania. Każdy ze scenariuszy badawczych odzwierciedlony został w formie tabeli.