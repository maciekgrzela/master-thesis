\chapter{Cel pracy}
Celem niniejszej pracy jest zaprojektowanie oraz implementacja systemu zarządzania dokumentami w przedsiębiorstwie. System ten, składa się z aplikacji webowej, wykorzystywanej przez pracowników przedsiębiorstwa wewnątrz firmy, oraz aplikacji mobilnej, wykorzystywanej przez wewnętrznych dostawców. Częściami składowymi aplikacji webowej są: interfejs wykorzystywany do zarządzania elementami systemu, a także serwer danych, z którym komunikować się można za pomocą napisanego API. Serwer ten, pełni także rolę źródła danych dla aplikacji mobilnej.

System umożliwia wprowadzanie dokumentów do bazy danych oraz przekazywanie ich pomiędzy pracownikami przedsiębiorstwa. Ponadto, program pozwala monitorować stan przekazywanych faktur, a także je archiwizować. Co więcej, aplikacja udostępnia pracownikowi szereg narzędzi związanych z określaniem trasy dostawy, wprowadzaniem oraz przeglądaniem notatek, powiadamianiem klienta o zbliżającej się dostawie, zarządzaniem kadrą pracowniczą czy też analizą statystyk.

Elementem dodatkowym przygotowanego rozwiązania jest wbudowany komunikator. Narzędzie to, pozwala na błyskawiczną wymianę informacji,  kluczowych dla procesu przetwarzania zamówienia przez członków zespołu pracowniczego.

Wyczerpujący opis zrealizowanego systemu, przedstawiony został w rozdziale \ref{}, a jego implementacja w rozdziale \ref{}.