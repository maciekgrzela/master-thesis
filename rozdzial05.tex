\chapter{Badania eksperymentalne}
\section{Uzyskane efekty pracy}
Z chwilą przystąpienia do realizacji oprogramowania, planowano stworzenie systemu zarządzania dokumentami w obrębie przedsiębiorstwa. System ten, miał składać się z aplikacji klienckich (tj. aplikacji webowej oraz mobilnej), a także interfejsu API.

W ramach strony webowej, użytkownik, po poprawnym uwierzytelnieniu oraz autoryzacji, powinien posiadać możliwości: przekazywania dokumentów, śledzenia ich aktualnego statusu, nadzorowania dostawców, przeglądania faktur archiwalnych, powiadamiania klientów o przetwarzaniu zamówienia, wyznaczania drogi dostawy towarów, prowadzenia wymiany wiadomości tekstowych z innymi użytkownikami, a także zarządzania zbiorami danych dotyczących konfiguracji systemu.

Ponadto, w obszarze aplikacji mobilnej, osoba korzystająca z programu, powinna móc: przekazywać dokumenty, wprowadzać uwagi dotyczące faktur, powiadamiać klienta o dostawie, wyznaczać trasę dostawy, wprowadzać wpisy dotyczące kilometrów pojazdu, definiować notatki handlowe, a także prowadzić konwersacje z wykorzystaniem komunikatora.

W kontekście interfejsu API, zaimplementowany miał zostać wydajny, a także bezpieczny system pozyskiwania oraz modyfikacji wszystkich danych, zawartych w ramach oprogramowania, posiadający mechanizmy uwierzytelniania oraz autoryzacji.

Wszystkie z przedstawionych powyżej założeń, zostały pomyślnie zaimplementowane w ramach stworzonego projektu. Dzięki doborowi odpowiednich narzędzi przed rozpoczęciem tworzenia oprogramowania, praca nad nim przebiegała w sposób systematyczny, a także efektywny. Podjęta decyzja, związana z wyborem bibliotek stylów Semantic-UI, a także Native-Base, odpowiednio dla aplikacji webowej oraz mobilnej, skutkowała powstaniem klarownych, łatwych w obsłudze i modułowych interfejsów użytkownika obu aplikacji.

Ponadto, wybór technologii .Net Core, jako środowiska dla interfejsu programowania aplikacji, przyczynił się do budowy skalowalnego, wydajnego i bezpiecznego interfejsu API, który może być z łatwością rozszerzany o nowe funkcjonalności.

Wszystkie z tworzonych elementów systemu, zostały sprawdzone w kontekście ewentualnych błędów w kodzie programu. 
\newpage
\section{Możliwości dalszego rozwoju}
Struktury i interfejsy graficzne aplikacji webowej oraz mobilnej, stworzone zostały w taki sposób, aby umożliwić łatwe wprowadzanie nowych funkcjonalności, w zależności od potrzeb użytkownika docelowego.

Jedną z takich potrzeb, może być moduł efektywnego pakowania pojazdu. W ramach tego rozwiązania, system określa które z dokumentów, w zależności od ich zbiorów produktów, mogą być zapakowane do pojazdu, aby maksymalnie wypełnić jego przestrzeń ładunkową. Z myślą o tej perspektywie, w modelu danych, zdefiniowane zostały atrybuty wymiarów powierzchni ładunkowej dla każdego z pojazdów.

Interesującą możliwością dalszego rozwoju oprogramowania jest implementacja algorytmu wyznaczania trasy dostawy, na podstawie informacji dotyczących ruchu drogowego. W przypadku takiego algorytmu, aplikacja identyfikuje ulice po których będzie poruszał się pojazd, a także pobiera dla nich informacje związane z aktualnym ruchem ulicznym, po czym uwzględnia je w kalkulacjach optymalnej trasy. Dzięki realizacji części serwerowej systemu, jako interfejs programowania aplikacji bazujący na punktach końcowych, proces tworzenia algorytmu może być realizowany w całkowicie odseparowanym środowisku. Po wykonaniu testów poprawności działania algorytmu, może on zostać w prosty sposób zintegrowany z aplikacją, poprzez wprowadzenie jego kodu wewnątrz endpointu.  

Dodatkowo, w przypadku wykorzystania oprogramowania w kontekście komercyjnym, zastosować można wybór płatnego dostawcy danych lokalizacyjnych (np. Google Maps), w celu zwiększenia ich wiarygodności oraz dokładności. Chcąc zastosować takie rozwiązanie w przygotowanym systemie, wystarczy wykupić usługę interfejsu lokalizowania od dostawcy, a następnie wprowadzić odpowiedni adres do tej usługi jako wartość właściwości dostawcy danych dla komponentów mapy.