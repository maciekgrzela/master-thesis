\chapter{Wstęp}
\section{Geneza pracy}
Usługi sieciowe, zarówno te dostępne publicznie jak i te realizowane dla celów prywatnych, pełnią kluczową rolę w kontekście funkcjonowania współczesnej sieci internetowej. Zapewne nikt z nas, nie jest w stanie wyobrazić sobie kształtu obecnego Internetu bez takich rozwiązań sieciowych jak obsługa poczty elektronicznej, realizacja transferu plików, czy też przede wszystkim dostęp do aplikacji oraz witryn internetowych. Szczególnie w obrębie ostatniej spośród wymienionych usług, na przestrzeni ostatnich lat zauważyć można bardzo dużą liczbę zmian dotyczących sposobu ich definiowania oraz realizacji. Powodem pojawiania się tych zmian, jest niewątpliwie konieczność zachowania bądź też zwiększenia poziomów wydajności, niezawodności oraz bezpieczeństwa oferowanych rozwiązań, uwzględniając coraz to większy ruch sieciowy, generowany przez nieustannie zwiększającą się liczbę użytkowników Internetu. Ponadto, od nowoczesnego systemu internetowego, wymaga się coraz to większego poziomu skalowalności, a także płynności działania.

Poparciem niniejszych słów, może być treść wydawanego w kilkuletnich odstępach czasu raportu firmy Cisco, dotyczącego przewidywań oraz trendów sieciowych (tj. Cisco Visual Networking Index). Zgodnie z przedstawionymi w przytoczonym raporcie informacjami, a także porównując informacje te, z faktycznymi wartościami wskaźników dotyczących ruchu w internecie, zaobserwować możemy niemalże trzykrotny wzrost globalnego ruchu sieciowego na przestrzeni ostatnich pięciu lat. Ponadto, liczba klienckich urządzeń sieciowych, wykorzystywanych w celu uzyskania dostępu do usług udostępnianych w Internecie, na przestrzeni analogicznego przedziału czasowego, zwiększyła się z wartości 2,4 urządzenia na osobę, do poziomu niemalże czterech hostów sieciowych przypadających na pojedynczego reprezentanta globalnej populacji.

Należy także zwrócić uwagę, jakiego typu ruch sieciowy pełni dominującą rolę w kontekście dzisiejszego Internetu. Ponad 80\% globalnego konsumenckiego ruchu internetowego stanowią dane dotyczące usług wideo, około dziesięciu procent światowego ruchu obejmują pozostałe treści udostępniane w ramach aplikacji oraz witryn internetowych, a pozostałe 10\% to ruch generowany m.in. przez usługi transferu plików, poczty elektronicznej, czy też gier online. Na podstawie tych informacji, zauważyć można, że ponad 90\% całości danych, przesyłanych w ramach globalnej sieci, musi być przetwarzanych przez aplikacje internetowe, bądź usługi sieciowe z nimi powiązane. Dlatego też, zaawansowane witryny internetowe komunikujące się z usługami sieciowymi, zwane dziś systemami internetowymi, tworzone są z wykorzystaniem coraz to bardziej udoskonalonych modeli architektonicznych, pozwalających na coraz to łatwiejszą budowę i rozwój rozwiązań przystosowanych do potrzeb aktualnego ruchu sieciowego.

Jednym z pierwszych, a także najbardziej podstawowych podejść do projektowania i implementacji systemów internetowych było wprowadzenie modelu architektury definiującego aplikacje monolityczne. W modelu tym, użytkownik aplikacji, wykorzystując oprogramowanie klienckie, którym w tym przypadku jest przeglądarka internetowa, wysyłał żądanie uzyskania zasobu definiując odpowiedni adres url (\textit{(ang. Uniform Resource Locator)}). Żądanie to, odwoływało się bezpośrednio do fizycznego zasobu zlokalizowanego na serwerze, który przed dostarczeniem do klienta był przetwarzany przez serwer w celu uzupełnienia go danymi uzyskanymi z zewnętrznych źródeł - m.in. z systemu bazodanowego. Odpowiednio przygotowana statyczna zawartość odpowiedzi serwera, przybierająca postać pliku HTML (\textit{ang. HyperText Markup Language}) była następnie przesyłana bezpośrednio do przeglądarki internetowej. Podejście to, wyróżniało się całkowitym brakiem dynamiki działania systemu internetowego, ponieważ każde zdarzenie wywoływane przez oprogramowanie klienta, wymagało zaadresowania i wygenerowania nowego żądania w kierunku serwera, którego odpowiedzią była nowa zawartość warstwy prezentacyjnej systemu.

W związku z zauważeniem pewnej regularności dotyczącej funkcjonowania większości systemów internetowych, związanej z faktem niejednokrotnego generowania nieznacznie różniących się od siebie odpowiedzi serwera, a także w związku z rozwojem języka skryptowego JavaScript oraz technologii Flash, aplikacje w ramach architektury monolitycznej zaczęły uwzględniać obsługę żądań zawierających przetworzone fragmenty warstwy prezentacyjnej. Ponadto, możliwa stała się dynamiczna podmiana określonych fragmentów treści, bez konieczności ponownego pozyskiwania pozostałej zawartości widoku. Usprawnienie to, opierające się na technice realizacji żądań asynchronicznych w ramach JavaScript (\textit{ang. AJAX - Asynchronous JavaScript and XML}) pozwoliło na poprawę wydajności działania aplikacji internetowych przyczyniając się do zmniejszenia częstotliwości generowania zapytań, a także redukcji rozmiaru pojedynczej odpowiedzi serwera. Rozwiązanie to, nie wpływało jednakże bezpośrednio na strukturę systemu, której głównymi mankamentami były: pojedynczy centralny punkt przetwarzania żądań, a także brak separacji logiki działania systemu od warstwy prezentacyjnej.

Niedoskonałości omówionego powyżej modelu zostały zniwelowane poprzez wprowadzenie architektury zorientowanej na serwisy (\textit{ang. SOA - Service Oriented Architecture}). W podejściu tym, dokonano separacji warstwy prezentacyjnej systemu, a także wszystkich pozostałych funkcjonalności dotyczących logiki biznesowej oraz przetwarzania danych. Reużywalne oraz autonomiczne usługi sieciowe pozwalały na realizację określonych funkcji systemu, a sposób komunikacji klienta z usługą, jak i komunikacji pomiędzy poszczególnymi serwisami definiowany był przez standaryzowane kontrakty. Zdefiniowanie architektury zorientowanej na serwisy umożliwiło budowę skalowalnych systemów internetowych, których poszczególne części mogły być realizowane w dowolnej technologii, a implementacja nowej funkcjonalności nie wymagała przebudowy pozostałych komponentów. Rozwiązanie to, wprowadzało jednak dodatkowy narzut dla każdej z przesyłanych wiadomości, wynikający ze ściśle określonej struktury żądania, tworzonej z wykorzystaniem języka XML (\textit{ang. Extensible Markup Language}). Ponadto, wraz ze wzrostem poziomu zaawansowania systemu internetowego, autonomiczoność oraz reużywalność poszczególnych komponentów malała ze względu na powstawanie specyficznych dla określonego rozwiązania zależności.

W związku z coraz to większymi wymaganiami dotyczącymi aplikacji internetowych, dominująca ówcześnie architektura rozproszonych usług sieciowych zastąpiona została poprzez model uwzględniający warstwę kliencką oraz interfejs programowania aplikacji (\textit{ang. API - Application Programming Interface}). W przypadku nowoczesnych systemów internetowych, oba z tych komponentów budowane są w oparciu o architekturę n-warstwową (\textit{ang. N-Tier Architecture Application}). W ramach niniejszego modelu, klient wysyła żądanie do interfejsu API, który na początku przetwarza jego treść, a następnie wywołuje usługę utworzoną w celu realizacji określonego zadania. Celem serwisu jest przetworzenie logiki biznesowej dla danej funkcjonalności, a także odwołanie się do usług dostępu do danych w celu ich uzyskania z zewnętrznego źródła informacji. Odpowiednio przygotowana odpowiedź jest następnie przekazywana do warstwy obsługi żądania, która zwraca ją określonemu klientowi. W przeciwieństwie do pierwszego z przytoczonych modeli, odpowiedzią API nie jest dokument HTML, a jedynie dane dotyczące zasobu, które chce uzyskać klient. Sam zasób natomiast, nie jest elementem warstwy prezentacji systemu a zbiorem danych lub typem operacji, które można na tym zbiorze wykonać. Upraszczając, stwierdzić można, że API pełni rolę pośrednika pomiędzy warstwą prezentacji a zbiorem danych oraz operacji ich przetwarzania, a także dostarczania. Poszczególne usługi realizujące logikę biznesową aplikacji zawarte są bezpośrednio wewnątrz API, co nie oznacza jednakże, że nie mogą odwoływać się do serwisów zewnętrznych. Takie podejście do budowania systemów internetowych zapewnia zarówno skalowalność poszczególnych aplikacji wchodzących w skład systemu, jak i rozwiązuje problemy architektury SOA związane z zależnościami występującymi pomiędzy usługami. Dlatego też, architektura ta jest powszechnie wykorzystywana w celu budowy i zarządzania nowoczesnymi oraz zaawansowanymi systemami internetowymi.

Zarówno zdecentralizowana architektura zorientowana na serwisy, jak i centralna architektura oparta o interfejs programowania aplikacji, w przeciwieństwie do architektury monolitycznej, dostarcza zdecydowanie więcej możliwości związanych z ewaluacją działania poszczególnych komponentów systemu. Dzięki powstaniu ostatnich dwóch, spośród trzech przedstawionych modeli architektonicznych, możliwe jest nie tylko zbudowanie efektywnie działającej aplikacji internetowej, ale także ciągła ocena poprawności implementacji jej komponentów, w celu ustawicznego doskonalenia całego systemu.

Niniejsza praca, traktować będzie o ewaluacji efektywności działania interfejsów programowania aplikacji, w kontekście jednych z dwóch najpopularniejszych środowisk rozwoju oraz uruchamiania api. Ponadto, porównane zostaną parametry wydajnościowe w kontekście określonych przypadków użycia interfejsu API, będącego niezbędną częścią powszechnie wykorzystywanej architektury systemów internetowych.
\section{Cel i zakres pracy}
Celem pracy jest porównanie wydajności działania interfejsów programowania aplikacji, tworzonych z wykorzystaniem języków programowania C\# oraz JavaScript. Interfejsy, wykonywane są w dwóch różnych środowiskach uruchomieniowych. Dla języka C\#, środowiskiem tym jest platforma .NET, natomiast dla języka JavaScript – platforma NodeJS. Analiza porównawcza, obejmować ma zarówno aspekty dotyczące efektywności działania samego interfejsu programowania aplikacji, jaki i rozwiązań wchodzących w skład tworzonego systemu. Wśród rozwiązań tych, wyróżnić należy mappery obiektowo-relacyjne, systemy bazodanowe, czy też mechanizmy zarządzania pamięcią podręczną. Niektóre spośród wymienionych elementów stanowią integralną część API, natomiast pozostałe służą do rozszerzenia jego funkcjonalności.

Zakres pracy obejmuje: przegląd literaturowy, implementację środowiska badawczego, realizację badań oraz opracowanie wyników. Przegląd literatury tyczy się aspektów związanych ze strukturą i zasadą działania interfejsów programowania aplikacji, a także kwestii dotyczących wykonywania pomiarów wydajności dla poszczególnych operacji sieciowych. Operacje sieciowe realizowane są w ramach obsługi żądania przez API. Etap implementacji środowisk badawczych składa się z budowy interfejsów w oparciu o porównywane środowiska rozwoju i uruchamiania aplikacji, a także konfiguracji platformy lokalnej oraz platform chmurowych, pozwalających na przeprowadzanie analizy działania systemów. Realizacja badań, przeprowadzona została pod kątem pomiaru czasu odpowiedzi na żądania użytkownika końcowego, biorąc pod uwagę aspekty: wywołania serii żądań, obsługi współbieżności procesów, dostępności zasobów platformy hostingowej, a także możliwości oferowanych przez mappery obiektowo-relacyjne oraz systemy bazodanowe. Celem etapu opracowania wyników jest przedstawienie, wizualizacja oraz analiza różnic wartości czasów odpowiedzi interfesjów API na poszczególne żądania, w odniesieniu do przeprowadzonych badań. Zastosowanymi kryteriami oceny podczas przeprowadzanej analizy jest czas odpowiedzi interfejsu programowania aplikacji dla wygenerowanego żądania, a także maksymalna liczba żądań jakie jest w stanie obsłużyć określone API. Przedstawione kryteria, uwzględniane będą w kontekście wykorzystywanego środowiska uruchomieniowego oraz technologii implementacyjnej. Przeprowadzone badania, mają służyć wskazaniu zarówno pozytywnych aspektów, jak i problemów dotyczących wydajności działania aplikacji tworzonych z wykorzystaniem porównywanych technologii. Ponadto, celem jest także przedstawienie możliwości zwiększenia efektywności tworzonych interfejsów programowania aplikacji.
\section{Struktura pracy}
Niniejsza praca, podzielona została na sześć rozdziałów.

Pierwszy z nich, napisany został w celu zobrazowania dziedziny rozważanego problemu, a także podkreślenia jego wagi w kontekście zagadnienia usług sieciowych. Ponadto, w rozdziale tym zdefiniowano cel popełnionej pracy oraz przedstawiono zakres czynności realizowanych w ramach przeprowadzonych badań.

W rozdziale drugim dokonano wprowadzenia teoretycznego do tematyki interfejsów programowania aplikacji oraz testowania usług sieciowych. Wprowadzenie to, w kontekście interfejsów API dotyczy zarówno struktury i zasady działania omawianej usługi sieciowej, jak i sposobu realizacji połączeń tej usługi z zewnętrznymi źródłami danych. W ramach wprowadzenia do tematyki testowania usług sieciowych wyjaśniono fundamentalne pojęcia teorii testowania oraz omówiono dostępne modele realizacji testów. Co więcej, nakreślono strategię wykonywania pomiarów wydajności w kontekście usług pracujących w sieciach komputerowych. W niniejszym rozdziale, zawarto również przegląd pozycji literaturowych, pomocnych w aspekcie realizacji badań, a także przegląd technologii informatycznych, wykorzystywanych w celu implementacji środowiska badawczego oraz wykonania pomiarów.

W ramach trzeciego z rozdziałów, zdefiniowano i omówiono każdy z aspektów problemu badawczego. Dzięki czemu, w kolejnej z sekcji pracy, możliwe było sformułowanie określonych scenariuszy badawczych.

Rozdział czwarty, stanowi o projekcie oraz implementacji środowiska badawczego, wykorzystywanego w celu dokonywania pomiarów...

Piąty z rozdziałów, ma na celu przedstawienie rezultatów wykonywanych badań. Rezultaty te, w ramach niniejszego rozdziału zostały zgrupowane względem zdefiniowanych uprzednio scenariuszy badawczych. Ponadto, dla uzyskanych pomiarów wykonano testy parametryczne, dzięki którym możliwa jest ocena istotności statystycznej zaobserwowanych różnic wynikowych. W ramach tego rozdziału, dokonano także analizy uzyskanych wyników.

Ostatni z rozdziałów pełni rolę podsumowania. Autor przedstawia w nim uzyskane efekty wykonanej pracy, a także nakreśla możliwości związane z dalszym rozwojem badań.